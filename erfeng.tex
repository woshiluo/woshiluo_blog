\documentclass[10pt]{beamer}
\usepackage[UTF8]{ctex}
% \usetheme{Warsaw}
\usecolortheme{dove}
\setbeamertemplate{background}{\includegraphics[height=\paperheight]{qwq.jpg}}
\usefonttheme[onlymath]{seri}
\usepackage{listings}
\usepackage{xcolor}
\usepackage{fontspec} % 定制字体
\newfontfamily\menlo{Fira Code}
\usepackage{xcolor} % 定制颜色
\definecolor{mygreen}{rgb}{0,0.6,0}
\definecolor{mygray}{rgb}{0.5,0.5,0.5}
\definecolor{mymauve}{rgb}{0.58,0,0.82}
\lstset{ %
	showtabs=false
backgroundcolor=\color{white},      % choose the background color
basicstyle=\footnotesize\ttfamily,  % size of fonts used for the code
columns=fullflexible,
tabsize=4,
breaklines=true,               % automatic line breaking only at whitespace
captionpos=b,                  % sets the caption-position to bottom
commentstyle=\color{mygreen},  % comment style
escapeinside={\%*}{*)},        % if you want to add LaTeX within your code
keywordstyle=\color{blue},     % keyword style
stringstyle=\color{mymauve}\ttfamily,  % string literal style
frame=single,
rulesepcolor=\color{red!20!green!20!blue!20},
% identifierstyle=\color{red},
language=c++,
}


\title{科学的二分及三分}
\author{Woshiluo\footnote{背景来源:恋×シンアイ彼女CG}}
\institute{My blog: \href{https://blog.woshiluo.com}{https://blog.woshiluo.com}}
\date{November 15, 2019}

\begin{document}
    \frame{\titlepage}
 
	\begin{frame}
		\frametitle{Content}
		\tableofcontents
    \end{frame}

	\section{序}

	\begin{frame}
		\frametitle{序}
		二分作为一种非常奇妙的东西,在 NOIP / CSP 考得较为平凡
		
		称二分为一种算法可能不如称二分为一种思想

		所以还是需要自己去做题理解的

		这篇课件仅仅讲述其思想及参考例题
	\end{frame}

	\section{二分}
	\subsection{普通二分}
	\begin{frame}
		\frametitle{普通二分}
		在数学上的二分法大家应该都有所了解

		至少我在学 $\sqrt{2}$ 的时候就有老师跟我们讲过

		不过这里要讲的二分会更加广义一些
	\end{frame}

	\begin{frame}[fragile]
		\frametitle{普通二分}
		通常来说,我们所讲述的二分都是指 「二分查找」

		即「在一个有序数组中查找某一元素的算法」

		\pause

		以最普通的有序数列中找数为例

		我们定义一个 \texttt{left} 和 \texttt{rig}

		求出 $ mid = \frac{left + rig}{2} $

		\pause

		判断 \texttt{mid} 与要查询的大小

		然后就可以缩小查找区间

		每次查询会将查询范围减半,所以复杂度为 $O(\log n)$

		\pause

		\begin{lstlisting}[title=参考代码, frame=shadowbox]
int left = 0, rig = 20000000001, res = 0;
while( left <= rig ) {
    int mid = ( left + rig ) >> 1;
    if( check(mid) < m )
        rig = mid - 1;
    else {
        res = mid;
        left = mid + 1;
    }
}\end{lstlisting}
	\end{frame}
	\begin{frame}
		\frametitle{普通二分}
		大家应该也注意到了上面的参考代码中的 \texttt{check()} 函数

		换一种思路,如果我们现在有一个问题,我们可以将问题化为 

		\pause

		有一个函数 $f(x)$ 在 $ left \leq x \leq rig $ 中 $f(x)$ 随 $x$ 增长而单增/单减

		\pause

		那么现在要找到 $ f(x) = y$ 时 $x$ 的值(保证 $ left \leq x \leq rig $ ) ,我们就可以使用二分的方式来解决
	\end{frame}
	\subsubsection{例题 Luogu P1873 砍树}
	\begin{frame}
		\frametitle{例题 Luogu P1873 砍树}
		题目链接: \href{https://www.luogu.com.cn/problem/P1873}{https://www.luogu.com.cn/problem/P1873}

		我们可以将题目中的问题设为 $f(x)$

		即当高度为 $x$ 时,我们能不能够获得足够的树木

		\pause

		能函数返回 $1$ 不能函数返回 $0$

		\pause

		现在要找到第一个 $1$ 所在的 $x$ 
	\end{frame}
	\begin{frame}
		\frametitle{例题 Luogu P1873 砍树}
		对于这一类问题,不同的人有不同的做法

		\pause

		在李煜东的「算法竞赛进阶指南」,详细的讲述了通过调整对于区间左右端点的偏移来解决问题的方案

		\pause

		这里讲述一种我比较习惯的方案,就是通过 \texttt{res} 的记录

		即每一次 \texttt{check()} 返回值为真时,我们记录下当前的 \texttt{res}

		最后输出 \texttt{res} 即可

		\pause

		参考代码: \href{https://www.luogu.com.cn/paste/uj4ayk68}{https://www.luogu.com.cn/paste/uj4ayk68}
	\end{frame}
	
	\subsubsection{附加习题}
	\begin{frame}
		\frametitle{附加习题}
		Luogu P5021 赛道修建

		参考代码: \href{https://www.luogu.com.cn/paste/u4ihzowj}{https://www.luogu.com.cn/paste/u4ihzowj}

		\pause

		Luogu P1084 疫情控制

		参考代码: \href{https://www.luogu.com.cn/paste/5vozgj4k}{https://www.luogu.com.cn/paste/5vozgj4k}

		后面会补博客的...
	\end{frame}

	\subsection{ 01 分数规划 }
	\begin{frame}
		\frametitle{01 分数规划}
		分数规划通常是指这一类问题

		每个物品有两个属性 $a_i$ $b_i$

		现在要选出若干个物品,使得 $\frac{\sum{b_i}}{\sum{a_i}}$ 最值
	\end{frame}

	\begin{frame}
		\frametitle{01 分数规划}
		设当前确定的答案为 $ans$
		
		假设我们要求的是最大的

		则

		\pause

		\begin{align}
			\nonumber
			\frac{\sum{b_i}}{\sum{a_i}} & \leq ans \\
			\nonumber
			\sum{b_i} & \leq \sum{a_i} \times ans \\
			\nonumber
			\sum{b_i} - \sum{a_i} \times ans  & \leq 0
		\end{align}

		\pause

		然后你发现这个式子是随左边的 $ans$ 变化而单调变化

		\pause

		二分即可
	\end{frame}

	\subsubsection{例题 NovaOJ 16037 地图}
	\begin{frame}
		\frametitle{例题 NovaOJ 16037 地图}
		这个题目很明显可以化成上面的式子

		\pause

		\texttt{check()}的时候,枚举起点,分别跑最长路,查询最长路是否 $ \geq 0 $ 

		然后就是标准的二分了

		复杂度 $ O( n \log n ) $

		\pause

		参考代码: \href{https://noj.ac/submission/2506}{https://noj.ac/submission/2506}
	\end{frame}
	\subsubsection{附加习题}
	\begin{frame}
		\frametitle{附加习题}

		Luogu P4322 [JSOI2016]最佳团体

		参考代码: \href{https://www.luogu.com.cn/paste/rgu4jfzq}{https://www.luogu.com.cn/paste/rgu4jfzq}
	\end{frame}

	\section{三分}
	\begin{frame}[fragile]
		\frametitle{三分}
		
		这个东西完全是食之无用,弃之可惜

		\pause

		因为这个东西是用于求单峰函数极值的

		\pause

		然而为什么不求导后直接二分呢?

		疑惑...

		和二分一个原理 

		\pause

		\begin{lstlisting}[title=参考代码, frame=shadowbox]
lmid = left + (right - left >> 1);
rmid = lmid + (right - lmid >> 1);  // 对右侧区间取半
if (cal(lmid) > cal(rmid))
    right = rmid;
else
    left = lmid;
\end{lstlisting}

	\end{frame}

	\section{尾}
	\begin{frame}
		\frametitle{尾}
		希望这篇可见能够给您一点帮助

		二分这个东西可以套各种东西

		比如数据结构,最短路,网络流,DP

		总而言之,不要被标准化的二分局限住了思想

		就这样子

		「Fightだよ!」
	\end{frame}

	\subsection{版权声明}

	\begin{frame}
		\frametitle{版权声明}
		本课件在写作过程中,参考了以下内容,在此向原内容创作者表示感谢

		\href{https://oi-wiki.org/basic/binary/}{ Oi - Wiki 二分 }

		\pause
		本课件使用 \href{https://creativecommons.org/licenses/by-sa/4.0/deed.zh}{CC BY-SA 4.0} 和 \href{https://github.com/zTrix/sata-license}{SATA} 协议
	\end{frame}

\end{document}
