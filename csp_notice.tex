\documentclass[10pt]{beamer}
\usepackage[UTF8]{ctex}
% \usetheme{Warsaw}
\usecolortheme{dove}
\setbeamertemplate{background}{\includegraphics[height=\paperheight]{bg.jpg}}

\title{ Woshiluo 的自我总结}
\author{Woshiluo\footnote{背景来源:恋×シンアイ彼女CG}}
\institute{My blog: \href{https://blog.woshiluo.com}{https://blog.woshiluo.com}}
\date{November 15, 2019}

\begin{document}
    \frame{\titlepage}
 
	\begin{frame}
		\frametitle{Content}
		\tableofcontents
    \end{frame}

	\section{序}

    \begin{frame}
        \frametitle{序}
        老师们说要我们这群参加过 NOI 的写写总结

		虽然我参加过 NOI,但是打铁了...

		而且感觉过了一年感觉自己更菜了

		不过还是写一写吧,就当自己的总结
    \end{frame}

	\section{心态}

	\subsection{考场心态}
	\begin{frame}
		\frametitle{考场心态}
		\sout{我应该是整个机房里最没有资格讨论考场心态的人}
		
		\pause

		首先就是考场不要慌

		NOI 系列比赛通常对你代码能力要求不大

		根据经验,你考场上敲不出来的东西

		绝大多数你的同学们也敲不出来

		\pause

		不要想结果

		\sout{比如 15owzLy1 是不是已经 AK 了啊之类的}

		想结果往往会分散注意力,不如好好想题目

		\sout{虽然这个东西说来容易做来难}
	\end{frame}

	\subsection{场外心态}
	\begin{frame}
		\frametitle{场外心态}
		众所周知,CSP-S2 是要考两天的

		\pause
		

		对于这种两天的考试,最好在 Day1 考完之后不要讨论题目一类的题目

		\sout{虽然 NOI 中间会强制社会活动}

		\pause
		

		然后就是不要自暴自弃
	\end{frame}

	\section{读题}
	\begin{frame}
		\frametitle{读题}
		\textbf{如果有能力,一定一定要手推样例}

		这个可以防止 90\% 的理解问题

		\pause
		

		最后将思路在演草纸上确定个大概

		再上手键盘

		可以最大程度的减少不必要的动脑和代码

		\pause
		

		还有,虽然说 CSP/NOIP 中确确实实有原题的状况出现\footnote{参见……算了,不鞭尸了}

		\pause
		

		但是一定不要觉得自己做过就上手瞎写,容易白给
	\end{frame}

	\section{程序}

	\subsection{代码以外的细节}
	\begin{frame}
		\frametitle{代码以外的细节\footnote{neta 「时间之外的往事」}}
		编译一定要加 \texttt{-Wall -Wextra -Wshadow},虽然会报一些无意义错误总是有用的

		\pause

		警惕 \texttt{Dev-CPP} 的默认头文件缓存,它可能会让你少加头文件

		\pause

		注意不要错误的在 \texttt{freopen()} 中打开文件

		\textbf{特别是写入 \texttt{.cpp/.ans} 这种奇妙的问题}

		\pause 

		不要沉迷优化暴力
	\end{frame}

	\subsection{代码中的奇妙细节}
	\begin{frame}
		\frametitle{代码中的奇妙细节}
		小心变量初值问题,不合适/未定义的初值可能样例并不能检查出来这种问提

		\sout{可以通过开 O2 来测试样例检测自己有没有} UB\footnote{未定义行为}
		
		\pause

		内存也是一个自己测不出来的东西,一定一定要算内存 desu

		\sout{我已经见过两个人没算内存死亡了}

		\pause

		虽然 STL 很快乐,而且 CCF 的机子也很快\footnote{至少去年是 \texttt{i7-8700k}}\footnote{但是去年也有被卡 STL 常的人}

		但是不开 O2 的 STL 因为封装过度,速度依然很感人

		\pause

		特判写 \texttt{namespace} 有很大的快乐

		不过注意分号有没有在特别的位置上
	\end{frame}

	\section{一点小东西}
	\subsection{思路}
	\begin{frame}
		\frametitle{思路}
		多上厕所,提神还能验证思路

		许多大佬们想题都是在上厕所时想起来的
		
		\sout{而且万一听到大佬在说正解呢?}

		\pause

		根据经验,第一直觉,往往是错的,不要死想

		\pause

		面对数学题的最好方法,就是打表!\footnote{neta 了……不说了,大家都知道}

		顺带一提,遇字符串不决,就用哈希

		\pause

		\sout{ $n^2$ 过百万,暴力碾表算} \footnote{参见 CCF NOI WC 2017 挑战}
	\end{frame}

	\subsection{场地}
	\begin{frame}
		\frametitle{场地}
		这次还是在七十中

		小心特派员的嗓门,真的很大,不要试图和她杠

		不要管别人在干什么,特别是不要怕键盘声音,他可能只是在调 \texttt{vimrc}
	\end{frame}

	\section{结}
	\begin{frame}
		\frametitle{结}
		嘛,但愿能帮助到大家吧

		总而言之,大家 RP ++

		「Fightだよ!」
	\end{frame}

\end{document}
